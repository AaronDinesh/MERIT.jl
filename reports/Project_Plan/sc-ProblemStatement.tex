\section*{Problem Statement}
There are 3 main pillars for any good software framework. It needs to be robust, easy to use and expand, and most importantly, open source. One additional requirement for this library is that it must run on a GPU. This allows for the processing of vast amounts of data, opening the space to more complex algorithms and higher resolution scans. The MERIT toolbox for MATLAB, written by Prof. Declan O'Loughlin et al, has laid the foundation on which other researchers can test imaging software as well as new algorithms. This library has already been used by multiple publications over the years with 13 citations attributed to the original paper and 16 forks of the repository. While MATLAB is easy to use and understand, it can be quite slow to execute. In order to write performant code, another language needs to be considered. Aruoba and Fernandez-Villaverde found, through testing, that MATLAB runs 3x slower than the C++ implementation and around 2.24x slower than the Julia implementation \cite{RN3}. Julia offers numerous benefits over both C++ and MATLAB. It can be considered a high-level language like MATLAB, while also offering a speed comparable to C++ due to the compilation of the code at run time. Various language features such as multiple dispatch also allow for easy extension of the code to accommodate for nuances in signal processing algorithms that may arise when dealing with scans from different parts of the body. Julia also offers a GPU compiler which allows Julia code to run on the GPU, allowing us to create function implementations that can use the full breadth of computing resources available.