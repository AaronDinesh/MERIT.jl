\section*{Introduction}
Microwave imaging has started to garner an interest amongst the medical community as an alternative and safer form of imaging when compared to more traditional methods such as X-rays. Recent clinical trials such as the “MARIA M4” system have proven that such microwave-based methods are more comfortable and offer a viable alternative to X-rays \cite{RN1}. Mammography is not the only area where this imaging modality is being trialed. It is seeing use in areas such as traumatic brain injury detection, bone degradation, and tumor detection \cite{RN2}. While the hardware has proved effective, the software side leaves a lot to be desired. Researchers often have to code their own data processing pipeline in order to get usable results for their studies. All this development subtracts from time that could be spent developing new algorithms and higher resolution systems. In their haste to complete a paper, bugs could be inadvertently introduced into the code, at best slowing down research while the bug is fixed, while at worst, it could bias the results without the researchers knowing. As such, good software needs to be built that would allow researchers to worry more about designing and testing algorithms rather than worrying about how to code the supporting functions that would allow them to test the effectiveness of these algorithms.